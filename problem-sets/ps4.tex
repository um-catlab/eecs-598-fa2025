\documentclass[12pt]{article}

%AMS-TeX packages
\usepackage{amssymb,amsmath,amsthm,stmaryrd}
%geometry (sets margin) and other useful packages
\usepackage[margin=1.25in]{geometry}
\usepackage{tikz}
\usepackage{tikz-cd}
\usepackage{graphicx,ctable,booktabs}
\usepackage{mathpartir}
\usepackage{minted}


\usepackage[sort&compress,square,comma,authoryear]{natbib}
\bibliographystyle{plainnat}

%
%Redefining sections as problems
%
\makeatletter
\newtheorem{definition}{Definition}
\newtheorem{lemma}{Lemma}
\newtheorem{theorem}{Theorem}
\newenvironment{problem}{\@startsection
       {section}
       {1}
       {-.2em}
       {-3.5ex plus -1ex minus -.2ex}
       {2.3ex plus .2ex}
       {\pagebreak[3]%forces pagebreak when space is small; use \eject for better results
       \large\bf\noindent{Problem }
       }
       }
       {%\vspace{1ex}\begin{center} \rule{0.3\linewidth}{.3pt}\end{center}}
       \begin{center}\large\bf \ldots\ldots\ldots\end{center}}
\makeatother


%
%Fancy-header package to modify header/page numbering
%
\usepackage{fancyhdr}
\pagestyle{fancy}
%\addtolength{\headwidth}{\marginparsep} %these change header-rule width
%\addtolength{\headwidth}{\marginparwidth}
%% \lhead{Problem \thesection}
\chead{}
\rhead{\thepage}
\lfoot{\small\scshape EECS 598: Category Theory}
\cfoot{}
\rfoot{\footnotesize PS 4}
\renewcommand{\headrulewidth}{.3pt}
\renewcommand{\footrulewidth}{.3pt}
\setlength\voffset{-0.25in}
\setlength\textheight{648pt}

%%%%%%%%%%%%%%%%%%%%%%%%%%%%%%%%%%%%%%%%%%%%%%%

%
%Contents of problem set
%

\newcommand{\meet}{\wedge}
\newcommand{\join}{\vee}
\newcommand{\iplmeets}{\mathrm{IPL}(\top,\meet)}
\newcommand{\iplneg}{\mathrm{IPL}(\top,\meet,\supset)}
\newcommand{\downset}{\mathcal P_{\downarrow}}
\newcommand{\down}{{\downarrow}}

\newcommand{\Set}{\mathrm{Set}}
\newcommand{\casePlus}[5]{\mathrm{case}_{+}\,{#1}\{\sigma_1{#2}\to {#3}|\sigma_2{#4}\to {#5}\}}
\newcommand{\caseZero}[1]{\mathrm{case}_0\,{#1}\{\}}
\newcommand{\id}{\mathrm{id}}
\newcommand{\lfpt}{\Sigma_{\mathrm{lfpt}}}

\newcommand{\cat}{\mathcal}

\newcommand{\stlccart}{\mathrm{STLC}(1,\times)}
\newcommand{\stlcccc}{\mathrm{STLC}(1,\times,\Rightarrow)}
\newcommand{\Tm}{\mathrm{Tm}}
\newcommand{\sem}[1]{\llbracket\cdot\rrbracket}

\begin{document}

\title{Problem Set 4}
\date{Released: October 9, 2025\\
  Updated Octover 22, 2025\\
  Due: October 23, 2025, 11:59pm
}
\maketitle

\textbf{Update}: Fixed typo in problem 2 part 1 and provide definition of composition of functors of SCwFs.

Submit your solutions to this homework on Canvas alone or in a group of 2.
Your solutions must be submitted in pdf produced using LaTeX.

\begin{problem}{Functor Comprehension Principle}
  Previously, you proved that if we have a specified product $A \times
  B$ for any two objects of $\mathcal C$ that there is a product
  functor $\mathcal C^2 \to \mathcal C$ that sends the pair $A,B$ to
  $A \times B$. This property can be generalized to any universal
  construction, as long as the universal property itself (i.e.,
  presheaf) is defined functorially.

  \begin{enumerate}
  \item Let $R : \mathcal C \to \mathcal P\mathcal D$ be a functor to presheaves on $\cat D$
    such that for every $A \in \mathcal C$, $R(A)$ is representable,
    i.e., we have a specified $F(A) : \mathcal D$ and natural
    isomorphism $i(A) : Y(F(A)) \cong R(A)$.

    Extend $F$ to a functor $\mathcal C \to \mathcal D$, i.e., define
    a functorial action on morphisms that takes $f : \mathcal C(A,B)$
    to $F(f) : \mathcal D(F(A), F(B))$ that preserves identity and
    composition.
  \end{enumerate}
\end{problem}

\begin{definition}
  A functor of SCwFs $F : \cat S \to \cat T$ consists of
  \begin{enumerate}
  \item A functor $F_c : \cat S_c \to \cat T_c$ of context/substitution categories.
  \item A function $F_t : \cat S_t \to \cat T_t$ on types.
  \item A natural transformation $F_{\Tm} : \Tm_{\cat S}(A) \to \Tm_{\cat T}(F_tA) \circ F_c^{\textrm{op}}$
  \end{enumerate}
  such that
  \begin{enumerate}
  \item $F_c$ preserves the terminal object up to isomorphism in that $F_c(\cdot)$ is terminal.
  \item $F_c$ preserves the context extension products in that the induced
    morphism $F_c(\Gamma \times A) \to F_c\Gamma \times F_t A$ is an isomorphism.
  \end{enumerate}

  Furthermore,
  \begin{enumerate}
  \item We say $F$ is \emph{faithful} if for every $\Gamma, A$, the
    function $F_{\Tm} : \Tm_{\cat S}(A)(\Gamma) \to \Tm_{\cat T}(F_tA)(F_c\Gamma)$
    is injective.
  \item If $S$ has a unit type, $1_S$, we say $F$ preserves the unit type
    if $\Tm_{\cat T}(F_t1_S) \cong \Tm_{\cat T}(1_T)$.
  \item If $S$ has product types $A \times B$, we say $F$ preserves
    product types if for every $A,B \in S_t$, that the induced natural
    transformation $\Tm_{\cat T}(F_t(A \times B)) \cong \Tm_{\cat T}(F_tA \times
    F_tB)$ is a natural isomorphism.
  \item If $S$ has function types $A \Rightarrow B$, we say $F$
    preserves function types if for every $A,B \in S_t$, that the
    induced natural transformation $\Tm_{\cat T}(F_t(A \Rightarrow B)) \to
    \Tm_{\cat T}(F_tA \Rightarrow F_tB)$ is an isomorphism.
  \item Given functors of SCwFs $F : \cat S \to \cat T$ and $G : \cat
    T \to \cat U$, their composition $G \circ F : \cat S \to \cat U$
    is defined as follows:
    \begin{enumerate}
    \item $(G \circ F)_c = G_c \circ F_c$
    \item $(G \circ F)_{t} = G_t \circ F_t$
    \item $(G \circ F)_{\Tm}(M) = G_{\Tm}(F_{\Tm}(M))$, which can be verified to satisfy naturality.
    \end{enumerate}
  \end{enumerate}
\end{definition}

For the next problem, you will need to use the initiality of STLC
semantics.
\begin{theorem}{Initiality of STLC Semantics (Simplified)}
  Let $\Sigma$ be an $\stlccart$ signature and let $\sigma$ be an
  interpretation of $\Sigma$ in a SCwF $\cat S$. Then
  \begin{enumerate}
  \item The semantics of $\stlccart$ in $\cat S$ constitutes a SCwF functor
    $\sem{\cdot} : \stlccart(\Sigma) \to \cat S$ that
    preserves product and unit types and agrees with $\sigma$ on base
    types and function symbols.
  \item If $F : \stlccart(\Sigma) \to \cat S$ is also a SCwF functor
    that preserves product and unit types as well as agreeing with
    $\sigma$ on base types and function symbols, then there is
    a\footnote{this natural isomorphism is unique among natural
    isomorphisms that respect the type structure in an appropriate
    sense, but for this problem set we need only its existence}
    natural isomorphism $\sem{\cdot}_c \cong F_c$.
  \end{enumerate}

  An analogous theorem holds for $\stlcccc(\Sigma)$ but where the
  functors $\sem{\cdot}, F$ additionally preserve function types .
\end{theorem}

\begin{problem}{Conservativity of Adding Function Types}
  Let $\Gamma \vdash M : A$ and $\Gamma \vdash N : B$ be terms in
  $\stlccart(\Sigma)$. Then we can view the same terms as terms
  in the extended language $\stlcccc(\Sigma)$. Does
  the addition of new proof terms change whether or not we can prove
  $M = N$?

  If $M = N$ is provable in the equational theory of
  $\stlccart(\Sigma)$, then clearly it is also provable in the
  extended logic $\stlcccc(\Sigma)$ since the latter logic contains
  all of the proof rules of the former. However, the converse is
  non-trivial to establish, and is called the \emph{conservativity} of
  the equational theory of $\stlcccc(\Sigma)$ over
  $\stlccart(\Sigma)$. It says that adding function types to our
  language doesn't allow us to prove any new theorems that could be
  formulated in the original logic. Compare the conservativity theorem
  we proved for IPL in Problem Set 1.

  Part 3 is quite difficult, so you may wish to finish the other parts
  of the construction before returning to it.
  \begin{enumerate}
  \item Show that if $F : \cat S \to \cat T$ and $G : \cat T \to \cat
    U$ are SCwF functors, and $G \circ F$ is faithful, then $F$ is
    faithful.
  \item Show that if $F, F' : \cat S \to \cat T$ are SCwF functors and
    $\alpha : F_c \cong F'_c$ is a natural isomorphism then $F$ is
    faithful if and only if $F'$ is faithful.
  \item (CHALLENGING) Show that for any category $\cat C$, the category of
    presheaves $\mathcal P\cat C$ is cartesian closed (HINT: the
    construction of exponentials is a direct generalization of the
    Heyting implication structure on downward closed sets in PS1).
  \item Define for any SCwF $\cat S$, a SCwF functor $Y : \cat S \to
    \textrm{democratic}(\mathcal P\cat {S}_c)$ that is faithful and
    preserves unit and product types.
  \item Define a SCwF functor $i : \stlccart(\Sigma) \to
    \stlcccc(\Sigma)$ that sends base types/function symbols to
    themselves and preserves product and unit types.
  \item Define a SCwF functor $j : \stlcccc(\Sigma) \to
    \textrm{democratic}(\mathcal P\cat {S}_c)$ that agrees with $Y :
    \stlccart(\Sigma) \to \textrm{democratic}(\mathcal P\cat {S}_c)$
    on base types/function symbols and preserves product and unit
    types.

  \item Conclude that $i$ is faithful\footnote{through a more complex
  construction, it can be established that $i$ is also full. See Crole
  chapter 4.10 for a similar proof.}, and therefore that adding
    function types to STLC is a conservative extension of the
    equational theory.
  \end{enumerate}
\end{problem}

\end{document}
