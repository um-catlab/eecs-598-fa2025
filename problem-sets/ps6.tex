\documentclass[12pt]{article}

%AMS-TeX packages
\usepackage{amssymb,amsmath,amsthm,stmaryrd}
%geometry (sets margin) and other useful packages
\usepackage[margin=1.25in]{geometry}
\usepackage{tikz}
\usepackage{tikz-cd}
\usepackage{graphicx,ctable,booktabs}
\usepackage{mathpartir}
\usepackage{minted}


\usepackage[sort&compress,square,comma,authoryear]{natbib}
\bibliographystyle{plainnat}

%
%Redefining sections as problems
%
\makeatletter
\newtheorem{definition}{Definition}
\newtheorem{lemma}{Lemma}
\newtheorem{theorem}{Theorem}
\newenvironment{problem}{\@startsection
       {section}
       {1}
       {-.2em}
       {-3.5ex plus -1ex minus -.2ex}
       {2.3ex plus .2ex}
       {\pagebreak[3]%forces pagebreak when space is small; use \eject for better results
       \large\bf\noindent{Problem }
       }
       }
       {%\vspace{1ex}\begin{center} \rule{0.3\linewidth}{.3pt}\end{center}}
       \begin{center}\large\bf \ldots\ldots\ldots\end{center}}
\makeatother


%
%Fancy-header package to modify header/page numbering
%
\usepackage{fancyhdr}
\pagestyle{fancy}
%\addtolength{\headwidth}{\marginparsep} %these change header-rule width
%\addtolength{\headwidth}{\marginparwidth}
%% \lhead{Problem \thesection}
\chead{}
\rhead{\thepage}
\lfoot{\small\scshape EECS 598: Category Theory}
\cfoot{}
\rfoot{\footnotesize PS 6}
\renewcommand{\headrulewidth}{.3pt}
\renewcommand{\footrulewidth}{.3pt}
\setlength\voffset{-0.25in}
\setlength\textheight{648pt}

%%%%%%%%%%%%%%%%%%%%%%%%%%%%%%%%%%%%%%%%%%%%%%%

%
%Contents of problem set
%

\newcommand{\rec}{\textrm{rec}}
\newcommand{\zero}{\textrm{zero}}
\newcommand{\suc}{\textrm{succ}}

\newcommand{\meet}{\wedge}
\newcommand{\join}{\vee}
\newcommand{\iplmeets}{\mathrm{IPL}(\top,\meet)}
\newcommand{\iplneg}{\mathrm{IPL}(\top,\meet,\supset)}
\newcommand{\downset}{\mathcal P_{\downarrow}}
\newcommand{\down}{{\downarrow}}

\newcommand{\Set}{\mathrm{Set}}
\newcommand{\casePlus}[5]{\mathrm{case}_{+}\,{#1}\{\sigma_1{#2}\to {#3}|\sigma_2{#4}\to {#5}\}}
\newcommand{\caseZero}[1]{\mathrm{case}_0\,{#1}\{\}}
\newcommand{\id}{\mathrm{id}}
\newcommand{\lfpt}{\Sigma_{\mathrm{lfpt}}}

\newcommand{\cat}{\mathcal}

\newcommand{\stlccart}{\mathrm{STLC}(1,\times)}
\newcommand{\stlcccc}{\mathrm{STLC}(1,\times,\Rightarrow)}
\newcommand{\Tm}{\mathrm{Tm}}
\newcommand{\sem}[1]{\llbracket\cdot\rrbracket}

\newcommand{\Timm}{T_{\textrm{imm}}}
\newcommand{\Tmut}{T_{\textrm{mut}}}

\begin{document}

\title{Problem Set 6: Adjunctions and Algebras}
\date{Released: November 13, 2025\\
  %% Updated Octover 22, 2025\\
  Due: Novemmber 25, 2025, 11:59pm
}
\maketitle

Submit your solutions to this homework on Canvas alone or in a group of 2.
Your solutions must be submitted in pdf produced using LaTeX.

\begin{definition}[Algebraic Theory of Immutable State]
  The algebraic theory of an immutable boolean state $\Timm$
  consists of one binary operation
  \begin{itemize}
  \item ``read'':  $r(x,y)$
  \end{itemize}
  and two laws
  \begin{itemize}
  \item Constancy law $r(x,x) = x$
  \item Diagonal law $r(r(x_{00},x_{01}), r(x_{10},x_{11})) = r(x_{00},x_{11})$
  \end{itemize}
\end{definition}

\begin{definition}[Algebraic Theory of Mutable State]
  The algebraic theory of mutable boolean state $\Tmut$ consists of two
  operations
  \begin{itemize}
  \item binary operation ``read'': $r(x,y)$
  \item two unary operations ``set 0'' $s_0(x)$ and ``set 1'' $s_1(x)$
  \end{itemize}
  Subject to the following laws
  \begin{itemize}
  \item Diagonal law: $r(r(x_{00},x_{01}), r(x_{10},x_{11})) = r(x_{00},x_{11})$
  \item Read-set: $r(s_0(x), s_1(x)) = x$
  \item Set-read: $s_i(r(x_0,x_1)) = s_i(x_i)$
  \item Set-set: $s_i(s_j(x)) = s_j(x)$
  \end{itemize}
\end{definition}

\begin{definition}[Free Algebra]
  A free $T$-algebra on a set $A$ for an algebraic theory $T$ consists of the following data:
  \begin{itemize}
  \item An algebra $F A$
  \item A function $\eta : A \to UFA$
  \item Such that for every algebra $Y$ pre-composition with $\eta$ is
    a bijection $\textrm{T-Alg}(FA,Y) \to \textrm{Set}(A, UB)$ from
    homomorphisms out of the free algebra on $A$ to functions into the
    underlying set of the algebra $B$
  \end{itemize}
\end{definition}

\begin{problem}{Relating properties of adjoint functors and their (co)-units}
  Prove Lemma 4.5.13 from Riehl's \emph{Category Theory in Context},
  page 140. Explain precisely how the case for unit follows from the
  case for co-unit (or vice-versa) by duality.
\end{problem}

\begin{problem}{Normal forms for computational algebras}
  \begin{enumerate}
  \item Let $A$ be a set. Define a $\Timm$-algebra structure whose underlying set is $A^2$ and show that it is the free $\Timm$-algebra on $A$.
  \item Show that for any $\Tmut$-algebra, the read operation satisfies the axioms of a $\Timm$-algebra.
  \item Let $A$ be a set. Define a $\Tmut$-algebra structure whose underlying set is $(2 \times A)^2$ and show that it is the free $\Tmut$-algebra on $A$.
  \end{enumerate}
\end{problem}

\end{document}
