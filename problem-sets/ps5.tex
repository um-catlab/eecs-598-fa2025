\documentclass[12pt]{article}

%AMS-TeX packages
\usepackage{amssymb,amsmath,amsthm,stmaryrd}
%geometry (sets margin) and other useful packages
\usepackage[margin=1.25in]{geometry}
\usepackage{tikz}
\usepackage{tikz-cd}
\usepackage{graphicx,ctable,booktabs}
\usepackage{mathpartir}
\usepackage{minted}


\usepackage[sort&compress,square,comma,authoryear]{natbib}
\bibliographystyle{plainnat}

%
%Redefining sections as problems
%
\makeatletter
\newtheorem{definition}{Definition}
\newtheorem{lemma}{Lemma}
\newtheorem{theorem}{Theorem}
\newenvironment{problem}{\@startsection
       {section}
       {1}
       {-.2em}
       {-3.5ex plus -1ex minus -.2ex}
       {2.3ex plus .2ex}
       {\pagebreak[3]%forces pagebreak when space is small; use \eject for better results
       \large\bf\noindent{Problem }
       }
       }
       {%\vspace{1ex}\begin{center} \rule{0.3\linewidth}{.3pt}\end{center}}
       \begin{center}\large\bf \ldots\ldots\ldots\end{center}}
\makeatother


%
%Fancy-header package to modify header/page numbering
%
\usepackage{fancyhdr}
\pagestyle{fancy}
%\addtolength{\headwidth}{\marginparsep} %these change header-rule width
%\addtolength{\headwidth}{\marginparwidth}
%% \lhead{Problem \thesection}
\chead{}
\rhead{\thepage}
\lfoot{\small\scshape EECS 598: Category Theory}
\cfoot{}
\rfoot{\footnotesize PS 5}
\renewcommand{\headrulewidth}{.3pt}
\renewcommand{\footrulewidth}{.3pt}
\setlength\voffset{-0.25in}
\setlength\textheight{648pt}

%%%%%%%%%%%%%%%%%%%%%%%%%%%%%%%%%%%%%%%%%%%%%%%

%
%Contents of problem set
%

\newcommand{\rec}{\textrm{rec}}
\newcommand{\zero}{\textrm{zero}}
\newcommand{\suc}{\textrm{succ}}

\newcommand{\meet}{\wedge}
\newcommand{\join}{\vee}
\newcommand{\iplmeets}{\mathrm{IPL}(\top,\meet)}
\newcommand{\iplneg}{\mathrm{IPL}(\top,\meet,\supset)}
\newcommand{\downset}{\mathcal P_{\downarrow}}
\newcommand{\down}{{\downarrow}}

\newcommand{\Set}{\mathrm{Set}}
\newcommand{\casePlus}[5]{\mathrm{case}_{+}\,{#1}\{\sigma_1{#2}\to {#3}|\sigma_2{#4}\to {#5}\}}
\newcommand{\caseZero}[1]{\mathrm{case}_0\,{#1}\{\}}
\newcommand{\id}{\mathrm{id}}
\newcommand{\lfpt}{\Sigma_{\mathrm{lfpt}}}

\newcommand{\cat}{\mathcal}

\newcommand{\stlccart}{\mathrm{STLC}(1,\times)}
\newcommand{\stlcccc}{\mathrm{STLC}(1,\times,\Rightarrow)}
\newcommand{\Tm}{\mathrm{Tm}}
\newcommand{\sem}[1]{\llbracket\cdot\rrbracket}

\begin{document}

\title{Problem Set 5}
\date{Released: October 30, 2025\\
  %% Updated Octover 22, 2025\\
  Due: Novemmber 13, 2025, 11:59pm
}
\maketitle

Submit your solutions to this homework on Canvas alone or in a group of 2.
Your solutions must be submitted in pdf produced using LaTeX.

\begin{definition}{Algebras, Initial Algebras}
  Let $F : \cat C \to \cat C$ be a functor.
  An $F$-algebra is a morphism $\alpha : F X \to X$.

  A homomorphism from $\alpha : F X \to X$ to $\beta : F Y \to Y$ is a
  morphism $\phi : X \to Y$ such that $\phi \circ \alpha = \beta \circ
  F$. Identity and composition of homomorphisms is given by identity
  and composition in $\cat C$. This defines a category $F$-Alg.

  An initial $F$-algebra is an initial object in $F$-Alg. Given an
  initial algebra $i : F(\mu F) \to \mu F$, and an algebra $\alpha : F
  X \to X$, we write $\textrm{rec}^F\alpha : \mu F \to X$ to mean the
  unique homomorphism from $i$ to $\alpha$.
\end{definition}

\begin{definition}
  Let $\cat C$ be a bicartesian category. A natural numbers object
  (NNO) in $\cat C$ is an initial algebra of the functor $F_{\mathbb
    N} X = 1 + X$.
\end{definition}


\begin{problem}{Programming with Peano}
  Let $\cat C$ be a cartesian closed category with a natural numbers
  object $i : F_{\mathbb N} N \to N$.

  \begin{itemize}
  \item Define a morphism $\textrm{add} : N \times N \to N$ that when
    $\cat C$ is the category of sets is the usual addition operation on
    natural numbers.
  \item Prove that zero is a left and right unit of $\textrm{add}$. That is
    \[ \textrm{add} \circ (\zero{}\circ {!}, \id_N) = \id_N : N \to N \]
    and
    \[ \textrm{add} \circ (\id_N, \zero{}\circ {!}) = \id_N : N \to N \]

    where $\zero{} : 1 \to N$ is $i \circ \sigma_1$.

    HINT: depending on how you define $\textrm{add}$, one of these two
    will be easy and one will require the uniqueness property of an NNO.
  \end{itemize}
\end{problem}

\begin{problem}{Generic Map-Fold Fusion}
  Let $\cat C$ be any category and $F, G : \cat C \to \cat C$ be
  functors.  Assume $i : F(\mu F) \to \mu F$ is an initial
  $F$-algebra, and $j : G(\mu G) \to G$ is an initial
  $G$-algebra. Then for any natural transformation $\tau : F
  \Rightarrow G$ and $G$-algebra $\alpha : G X \to X$, we can
  construct a ``pipeline'' $\textrm{rec}^G\alpha \circ \textrm{rec}^F
  (j \circ \tau_{\mu G})$. This pipeline first ``maps'', transforming
  the $F$ nodes of the $\mu F$-tree into $G$ nodes, constructing a
  $\mu G$-tree. Then we perform a ``fold'', reducing the $\mu G$ tree
  into a single $X$ value. 

  Prove that any such pipeline can be fused into a single fold:
  construct an $F$-algebra $\beta : FX \to X$ such that
  \[ \textrm{rec}^F \beta = \textrm{rec}^G\alpha \circ \textrm{rec}^F (j \circ \tau_{\mu G})\]
\end{problem}

\end{document}
