\documentclass[12pt]{article}

%AMS-TeX packages

\usepackage{amssymb,amsmath,amsthm}
\usepackage{tikz-cd}
% geometry (sets margin) and other useful packages
\usepackage[margin=1.25in]{geometry}
\usepackage{graphicx,ctable,booktabs}

\usepackage[sort&compress,square,comma,authoryear]{natbib}
\bibliographystyle{plainnat}

\newtheorem{theorem}{Theorem}
\newtheorem{lemma}{Lemma}
\newtheorem{corollary}{Corollary}
\newtheorem{definition}{Definition}

\newcommand{\self}{\mathrm{self}}
\newcommand{\tm}{\mathrm{Tm}}
\newcommand{\op}{\mathrm{op}}
\newcommand{\eval}{\mathrm{eval}}
\newcommand{\Yon}{\mathrm{Yoneda }}
\newcommand{\Set}{\mathrm{Set}}
\newcommand{\dem}{\mathrm{dem}}
\newcommand{\cont}{\mathrm{Cont}}
\newcommand{\pow}{\mathscr P}
\newcommand{\sem}[1]{\llbracket#1\rrbracket}
\newcommand{\semprod}[1]{\llbracket#1\rrbracket}
\newcommand{\semccc}[1]{\llparenthesis#1\rrparenthesis}
\newcommand{\sole}{\mathrm{sole}}
\newcommand{\var}{\mathrm{var}}
\newcommand{\app}{\mathrm{app}}
\newcommand{\Un}[1]{\mathrm{Un}_{#1}}

%
%Fancy-header package to modify header/page numbering
%
\usepackage{fancyhdr}
\pagestyle{fancy}
%\addtolength{\headwidth}{\marginparsep} %these change header-rule width
%\addtolength{\headwidth}{\marginparwidth}
\lhead{Section \thesection}
\chead{}
\rhead{\thepage}
\lfoot{\small\scshape EECS 598: Category Theory}
\cfoot{}
\rfoot{\footnotesize Scribed Notes}
\renewcommand{\headrulewidth}{.3pt}
\renewcommand{\footrulewidth}{.3pt}
\setlength\voffset{-0.25in}
\setlength\textheight{648pt}

%%%%%%%%%%%%%%%%%%%%%%%%%%%%%%%%%%%%%%%%%%%%%%%
\begin{document}

\title{Lecture 12: Initiality of STLC}
\author{Lecturer: Max S. New\\
  Scribe: Conner Rose}
\date{October 6, 2025}
\maketitle

Things we need to remember about presheafs:

\begin{enumerate}
  \item A presheaf is a functor $P : \mathcal{C}^{\op} \to \mathrm{Set}$
  \item A presheaf \emph{is} a universal property
  \item An object $x \in \mathcal{C}$ ``has'' the universal property $P$ if
        \[
          \Yon x \cong P
        \]
  \item Equivalently, given a universal element $\eta : P(x)$, then
        $\eta \circ_{p} - : \Yon x \to P$ is a natural isomorphism
\end{enumerate}

\section{Weak Initiality of IPL($\Sigma$)}

\begin{enumerate}
  \item IPL($\Sigma$) form a biHeyting (pre)algebra, ``tautological
        interpretation'' is self of $\Sigma$ in IPL($\Sigma$).
  \item For all interpretations $\sigma$ in a biHeyting prealgebra $P$,
        \begin{enumerate}
          \item $[\![\cdot]\!] : \text{IPL}(\Sigma) \to P$
                \begin{enumerate}
                  \item monotone
                  \item preserves biHeyting algebra structure; sends meets to meets, joins to joins, etc.
                  \item preserves the interpretation
                \end{enumerate}
          \item We have a unique such function (up to $\leq, \geq$)
        \end{enumerate}
\end{enumerate}

% syntax | sound + complete model | typical model
% STLC(0, +, 1 x, =>) | simple category w/ families +(0, +, 1, x, =>) | biCartesian Closed Category
% STLC($\emptyset$) | Simple category w/ families | Cartesian Categories
% STLC(1, x, =>) | SCwF(1, x, =>) | Cartesian Closed Category
% STLC(0, +, 1, x) | SCwF(0, +, 1, x) | Distributive biCartesian Category

\begin{tabular}{|l|l|l|}
  \hline
  \textbf{Syntax}                   & \textbf{Sound \& Complete Model}  & \textbf{Typical Model}            \\
  \hline
  STLC($0, +, 1, \times, \implies$) & SCwF($0, +, 1, \times, \implies$) & BiCartesian Closed Category       \\
  \hline
  STLC($\emptyset$)                 & SCwF                              & Cartesian Categories              \\
  \hline
  STLC($1, \times, \implies$)       & SCwF($1, \times, \implies$)       & Cartesian Closed Category         \\
  \hline
  STLC($0, +, 1, \times$)           & SCwF($0, +, 1, \times$)           & Distributive BiCartesian Category \\
  \hline
  STLC($1, \times$)                 & SCwF($1, \times$)                 & Cartesian Categories              \\
  \hline
\end{tabular}

\begin{definition}[biCartesian closed cateogry]
  A biCartesian closed cateogry is a category $\mathcal{C}$ with
  $\forall A, B \in \mathcal{C}$ we have
  \begin{itemize}
    \item a coproduct, $A + B$, along with inclusion maps $\sigma_1, \sigma_2$
    \item a product $A \times B$, along with projection maps $\pi_1, \pi_2$
    \item exponentials $A \implies B$, along with an evaluation map, $\eval : B^A \times A \to B$
  \end{itemize}
  Initial and terminal objects 1, 0.
  \begin{itemize}
    \item Cartesian: 1, $A \times B$
    \item coCartesian: 0, $A + B$
    \item Cartesian closed: 1, $A \times B, A \implies B$
  \end{itemize}
\end{definition}

\begin{definition}[Distributive coproducts, initial objects]
  Let $\mathcal{C}$ be a Cartesian category. A coproduct $A \times B$ in
  $\mathcal{C}$ is distributive when $\forall C$, we have
  \[
    (A + B) \times C \xleftarrow{\sim} (A \times C) + (B \times C)
  \]
  as an isomorphism.
  An initial object is given when
  \[
    0 \times C \xleftarrow{\sim} 0
  \]
  is an isomorphism.
\end{definition}

Coproduct is a ``left-handed'' universal property, meaning it's easy to map
\emph{out} of it. We can simply handle cases.

\begin{theorem}
  Every biCartesian closed category is distributive.
\end{theorem}
\begin{proof}
  First, let's show that $0 \times C \xleftarrow{\sim} 0$ is an isomorphism.
  The Yoneda embedding is fully-faithful, so, it suffices to show that
  $\Yon^\text{op}(0 \times C) \xleftarrow{\sim} \Yon^\text{op} 0$.
  For all $X \in \mathcal{C}$, we have
  \begin{align*}
    \mathcal{C}(0 \times C, X) & \cong  \mathcal{C}(0, X^C) \\
                               & \cong  1                   \\
                               & \cong  \mathcal{C}(0, X)   \\
  \end{align*}
  Now, to show that $(A + B) \times C \xleftarrow{\sim} (A \times C) + (B \times C)$
  an isomorphism, we have
  \begin{align*}
    \mathcal{C}((A + B) \times C, X) & \cong \mathcal{C}(A + B, X^C)                                      \\
                                     & \cong \mathcal{C}(A, X^C) \times \mathcal{C}(B, X^C)               \\
                                     & \cong \mathcal{C}(A \times C, X) \times \mathcal{C}(B \times C, X) \\
                                     & \cong \mathcal{C}((A \times C) + (B \times C), X)
  \end{align*}
\end{proof}
\begin{definition}[Simple Category with Families]
  A simple category with families (SCwF), $S$ consists of
  \begin{enumerate}
    \item A set $S_t$ of ``types''
    \item A category $S_C$ of ``contexts and substitutions''
    \item For every $A \in S_t$, we have a ``presheaf of terms'', $\tm(A)$ on $S_C$
    \item $S_C$ has a terminal object $\bullet \in S_C$
    \item $\forall \Gamma \in S_C, A \in S_t$, we have a ``product context'',
          $\Gamma \times A$ such that $\text{Yoneda}(\Gamma \times A) \cong \text{Yoneda }\Gamma \times Tm(A)$.
          I.e., $S_C(\Delta, \Gamma \times A) \cong S_C(\Delta, \Gamma) \times \tm(A)\Delta$.
  \end{enumerate}
\end{definition}
Let us define a SCwF called STLC($\Sigma$)($\dots (\text{connectives})$)
\begin{enumerate}
  \item Types are types of STLC
  \item Contexts are (syntactic) contexts $\Gamma$. As in PS2,
        for $\gamma : \Delta \to \Gamma$, and
        $\forall(x : A) \in \Gamma$, we have $\Delta \vdash \gamma(x) : A$.
  \item $\tm(A)(\Gamma) := \{ M \mid \Gamma \vdash M : A\}$,
        for $M \in \tm(A)(\Gamma)$ and $\gamma : \Delta \to \Gamma$, we have presheaf action
        $M \circ \gamma := M[\gamma]$
  \item We have the terminal context $\Gamma \to \bullet \cong 1$.
  \item Let $\Gamma, A$. We want to construct a context, $\Gamma \times A$, such that
        $(\Delta \to \Gamma \times A) \cong (\Delta \to \Gamma) \times (\Delta \vdash \bullet : A)$.
        We can define $\Gamma \times A := \Gamma, x \in A$ be extending $\Gamma$ with a free
        variable of the type $A$. Given $\gamma : \Delta \to \Gamma$ and $\Delta \vdash M : A$,
        we have $\gamma, M / x : \Delta \to \Gamma, x \in A$.
\end{enumerate}

Let $\mathcal{C}$ be a Cartesian category. We will define a SCwF called
$\dem(\mathcal{C})$.
\begin{enumerate}
  \item Types are objects of $\mathcal{C}$
  \item $\dem(\mathcal{C})_C := \mathcal{C}$
  \item $\forall A \in C_0$ we have $\dem(\tm(A)) = \Yon(A) : Psh(\mathcal{C})$
  \item Terminal object, free
  \item $\Gamma \times A$, free
\end{enumerate}

Fix a SCwF $S$.
\begin{enumerate}
  \item For $A, B \in S_t$, we define a product type $A \times B \in S_t$
        with $\tm_S(A \times B) \cong \tm_S A \times \tm_S B$.
        I.e., $\forall \Gamma$, we have $\tm_S(A \times B)(\Gamma) \cong (\tm_S A)(\Gamma) \times(\tm_S B)(\Gamma)$
\end{enumerate}
\begin{theorem}
  $\dem(\mathcal{C})$ always has product types $A \times B$, which are simply given by their product
  in $\mathcal{C}$.
\end{theorem}
\begin{theorem}
  STLC($\cdots$, $\times$) has products
  \[
    \Gamma \vdash \bullet : A \times B \cong (\Gamma \vdash \bullet : A) \times (\Gamma \vdash \bullet : B)
  \]
  Right to left corresponds to the product introduction rule, i.e., $(M, N)[\gamma] = (M[\gamma], N[\gamma])$.
  Left to right corresponds to the product elimination rules, i.e., $(\pi_i
    M)[\gamma] = \pi_i(M[\gamma])$.

  A unit type in $S$ is a type $1 \in S_t$. We have
  \begin{align*}
    \tm 1           & \cong 1     \\
    (\tm 1)(\Gamma) & \cong \{+\}
  \end{align*}

  A function type $A \Rightarrow B \in S$, we have $\tm(A \Rightarrow B)(\Gamma)
    \cong \tm(B)(\Gamma \times A)$
\end{theorem}
\begin{theorem}
  If STLC has function types in the syntactic sense then SCwF STLC has function
  types in the semantic sense. I.e., $(Mx)[\gamma, x/x] = M[\gamma]x$
\end{theorem}
In a SCwF $S$ with types $A, C$, we have a ``continuation presheaf'', $\cont A B$ on $S_c$
with $(\cont A B)(\Gamma) := (\tm B)(\Gamma \times A)$.
We have an ``empty type''' 0 in a SCwF is a type $0 \in S_t$ with $(\cont 0 C) \cong 1$
for all $C \in \mathcal{S_t}$.
For $A, B \in S_t$, we have a ``sum type''' $A + B \in S_t$ such that for all $C \in S_t$, we have
$\cont(A + B)C \cong \cont AC \times \cont BC$

For $\dem(\mathcal{C})$, having an empty type is equivalent to having a
distributive initial object. For all $C \in S_t$, we have $\cont 0 C \cong 1$,
so for all $\Gamma, C$ we have $\mathcal{C}(\Gamma \times 0, C) \cong 1 \cong
  \mathcal{C}(0, C)$, so $\Gamma \times 0 \cong 0$. Similarly, sum types
correspond to a distributive coproduct.
\end{document}
